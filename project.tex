\documentclass[12pt,abstracton]{scrartcl}
%packages
\usepackage{bbm}
\usepackage{amssymb}
\usepackage{amsthm}
\usepackage{latexsym}
\usepackage{amsfonts}
\usepackage{amsmath}
\usepackage{texdraw}
\usepackage{euscript}
\usepackage{graphicx}
\usepackage{multirow}
\usepackage{mathrsfs}
\usepackage{textcomp}
\usepackage{color}
\usepackage{fancyvrb}
\usepackage{pstricks,pst-node,pst-tree}
\usepackage{subfig}
\usepackage{tikz}
\usepackage{url}
%margins
\setlength{\textwidth}{6.5in} \setlength{\evensidemargin}{0in}
\setlength{\oddsidemargin}{0in}
\setlength{\textheight}{9in}
%section start markers
\newcommand{\lem}{\paragraph*{Lemma:}}
\newcommand{\cor}{\paragraph*{Corollary:}}
\newcommand{\prb}[1]{\section*{Problem {\text{#1}:}}}
\newcommand{\sprb}[1]{\subsection*{Item {\text{#1}:}}}
%abbreviations
\newcommand{\fa}{\forall}
\newcommand{\EE}{\exists}
\newcommand{\ee}{\epsilon}
\newcommand{\dd}{\delta}
\newcommand{\rr}{\mathbb{R}}
\newcommand{\rrx}{\rr^{\times}}
\newcommand{\qq}{\mathbb{Q}}
\newcommand{\nn}{\mathbb{N}}
\newcommand{\zz}{\mathbb{Z}}
\newcommand{\cc}{\mathbb{C}}
\newcommand{\ccx}{\cc^{\times}}
\newcommand{\oo}{\mathbb{O}}
\newcommand{\hh}{\mathbb{H}}
\newcommand{\pp}{\mathbb{P}}
\newcommand{\prm}[1]{#1^{\prime}}
\newcommand{\Letf}[3]{Let \({#1}:{#2}\rightarrow{#3}\) be}
\newcommand{\pard}[2]{\frac{\partial {#1}}{\partial {#2}}}
\newcommand{\FIT}{First Isomorphism Theorem}
\newcommand{\img}{\mathrm{img}\ }
\newcommand{\eq}[1]{\begin{equation}#1\end{equation}}
\newcommand{\ceil}[1]{\left\lceil#1\right\rceil}
\newcommand{\floor}[1]{\left\lfloor#1\right\rfloor}
\newcommand{\ip}[2]{\langle{#1},{#2}\rangle}
\newcommand{\contra}{\Rightarrow\Leftarrow}
\newtheorem{thm}{Theorem}
\newtheorem{clm}{Claim}
%section end markers
\newcommand{\Qed}{\hfill$\square$\newline\newline}
\newcommand{\QED}{\hfill$\square$}
%\renewcommand*\arraystretch{1.1}
\renewcommand*{\thefootnote}{\arabic{footnote}}

\title{Identifying Patterns in Expert ML Code\footnote{Pun intended}}
\subtitle{Comp150FP Final Project\footnote{Submitted in partial fulfillment of the requirements for M.S. in computer science}}
\author{Zhe Lu}
\date{10 December 2013}

\begin{document}
\bibliographystyle{alpha}
\input{epsf.sty}
\maketitle
\begin{abstract}
The aim of this paper is two-fold: (1)~identify features common to code written in Standard ML by experts and
(2)~evaluate how well expert ML code conforms to published style guides for ML.
\end{abstract}
\section{INTRODUCTION}
There are many ways to define what an expert is.
An often cited rule attributed to Malcolm Gladwell is that one requires 10000 hours of deliberate practice
in a particular field to become an expert \cite{Gla08}. But if we want to provide novice programmers
with specifics during these 10000 hours, what should we tell them? One challenge is that experts themselves
do not necessarily know what features identify them as experts and what features are unique to them as individuals.

I attempt to shed light onto this project by employing software metrics, by trying to find measureable quantities
that provide some metric for evaluating software. Specifically,
I approach this problem by examining source code written by experts. Source code analysis has been a topic of interest.

For samples of expert code, I turned to prominent projects where the source code is readily available:
\begin{enumerate}
\item The SMLNJ compiler,\footnote{\url{http://smlnj.cs.uchicago.edu/dist/working/110.76/compiler.tgz}} whose parser was used extensively for code analysis in this project
\item The MLKit compiler for SML\footnote{\url{http://www.itu.dk/research/mlkit/dist/mlkit-4.3.0-src.tgz}}
\item FoxNet, an implementation of the standard TCP/IP networking protocol stack in SML\footnote{\url{ftp://ftp.cs.cmu.edu/project/fox/snapshot/foxnet.tar.gz}}
\item Til, a type-directed compiler for ML\footnote{\url{http://www.cs.cornell.edu/Info/People/jgm/til.tar.Z}}
\item MLton, an open-source, whole-program, optimizing SML compiler\footnote{\url{http://sourceforge.net/projects/mlton/files/mlton/20130715/mlton-20130715.src.tgz/download}}
\end{enumerate}

The layout of this paper is as follow.
Section~\ref{sec:metric} provides a very brief background
concerning software metrics and source code analysis.
I look at how well expert code conforms to readily available
style guides in section~\ref{sec:style}.
Section~\ref{sec:syntax} discusses an analysis
of the distribution of syntactic forms. The implementation is discussed
in section~\ref{sec:impl}. Finally, I conclude and reflect in section~\ref{sec:future}.
\section{SOFTWARE METRICS AND SOURCE CODE ANALYSIS}\label{sec:metric}
A definition from Binkley is \cite{Bin07}:
\begin{quote}
\emph{Source code analysis} is the process of extracting
information about a program from its source code
or artifacts (e.g., from Java byte code or execution
traces) generated from the source code using automatic tools.
\emph{Source code} is any static, textual,
human readable, fully executable description of
a computer program that can be compiled automatically into an executable form.
\end{quote}
\section{STYLE}\label{sec:style}
\subsection{Tabs vs. Spaces}
\subsection{The 80-Column Rule}
\subsection{The Off-Side Rule}
\subsection{Misuse in Expressions}
\subsection{Newlines and Grouping}
\section{SYNTACTIC FORMS}\label{sec:syntax}
\subsection{Imperative Features}\label{subsec:imper}
In Prof. Norman Ramsey's class on programming languages at Tufts University, one easy way to lose points
on a given assignment was the unnecessary use \emph{imperative} language features when programming in a \emph{functional} language.
In Standard ML, there are two main imperative constructs:
(1)~the while expression and (2)~a list of expression separated by semicolons (similar to the \texttt{begin...end} syntax of Pascal)\cite{Ull98}.
\subsection{Common Library Functions}\label{subsec:imp}
\subsection{User-Defined Functions}\label{subsec:userfun}
\subsection{Program Structure}\label{subsec:struct}
\section{IMPLEMENTATION}\label{sec:impl}
\section{CONCLUSIONS AND FUTURE WORK}\label{sec:future}
\section*{Acknowledgments}
\begin{thebibliography}{99}
\bibitem[Gladwell 2008]{Gla08} Gladwell, M. (2008) \emph{Outliers}. Little, Brown and Co., New York, NY.
\bibitem[Ullman 1998]{Ull98} Ullman, J.\,D. (1998) \emph{Elements of ML Programming} (ML97 Ed). Prentice Hall, Upper Saddle River, NJ.
\bibitem[Binkley 2007]{Bin07} Binkley, D. (2007) Source Code Analysis: A Road Map. In \emph{Proceedings of the Interalnational Conference on Software Engineering (ICSE'07), Workshop on the Future of Software Engineering (FOSE'07)}. 104--119. DOI:\url{http://dx.doi.org/10.1109/FOSE.2007.27}
\bibitem[Vogelsang et al. 2010]{Vog10} Vogelsang, S., Fehnker, A., Huuck, R., Reif, W. (2010) Software Metrics in Static Program Analysis. In \emph{Formal Methods and Software Engineering. Proceedings 12th International Conference on Formal Engineering Methods (ICFEM'10)}. 485--500.
\bibitem[Ayerbe et al. 1998]{Aye98} Ayerbe, A., Vazquez, I. (1998) Software Products Quality Improvement with a Programming Style Guide and a Measurement Process. In \emph{Proceedings
22nd Annual International Computer Software and Applications Conference (COMPSAC'98)}. 172--178. DOI:\url{http://dx.doi.org/10.1109/WPC.2004.1311066}
\end{thebibliography}
\end{document}
